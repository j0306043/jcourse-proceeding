\documentclass[autodetect-engine,dvi=dvipdfmx,ja=standard,twocolumn,jbase=13.35Q]{bxjsarticle} % 日本語, LaTeXエンジン自動判定, 欧文9.5pt=和文13.35Q
%\documentclass[twocolumn,dvipdfmx]{jsarticle} % (古いので廃止) pLaTeX+dvipdfmxユーザ用
%\documentclass[a4paper,twocolumn]{article} % for English only manuscript

%%%%%%%%%%%%%%%%%%%%%%%%%%%%%%%%%%%%%%%%%%%%%%%%%%%%%%%%%
%% <local definitions here>

\usepackage{latexsym}
\usepackage{graphicx}
\usepackage{amssymb,amsmath}
\usepackage{url}
%% 本文と数式のフォントをTimes(注:New Romanではない)互換フォントにする。
%% AMS(数学)関係のパッケージのusepackageよりも後に置くこと。(前に置くと、
%% AMS関係のパッケージをusepackageしたところでエラーになるので注意)
\usepackage{newtxtext}
\usepackage[varg]{newtxmath}
%% PDFのメタデータにTitleとAuthorを入れる。
\usepackage[pdfusetitle]{hyperref}
%% 行間を詰める。
\renewcommand{\baselinestretch}{0.9}
%% itemize, enumerate, descriptionに追加される行間をなくす。
% \usepackage{enumitem}
% \setlist{topsep=0pt,parsep=0pt,partopsep=0pt,itemsep=0pt}

%% </local definitions here>
%%%%%%%%%%%%%%%%%%%%%%%%%%%%%%%%%%%%%%%%%%%%%%%%%%%%%%%%%

%--------卒業研究予稿集用スタイルファイルのパラメタ ここから--------
\usepackage{jcourse-proceeding}
\jcpnumber{A--2--1} % 発表番号
\title{卒業研究報告予稿集用の{\LaTeX}スタイルファイル} % 卒業研究の題目
%\jcpsubtitle{(タイトルの下に付加情報を付けられます。太字。中央配置。サブタイトルなど。例→) 良かったら卒業研究以外にも使ってください}
\author{岩手 太郎} % 発表者氏名
\jcplaboratory{(岩手研究室)} % 研究室名
%\jcpunderauthor{(著者の下に付加情報を付けられます。普通字。左寄せ。タイトル領域で特に言及したい文献情報など。例→) T. Iwate, H. Morioka, and G. Ueda, ``This is a sample title of an article in some journal or conference,'' \emph{International Journal of Something}, Vol.3, No.11, pp.110--118, Jan. 2020. }

\jcpabstract{
本稿(sample.tex)は、卒業研究報告予稿集原稿のフォーマットおよび
{\LaTeX}用の卒業研究報告用スタイルファイル(jcource-proceeding.sty)の
使用法を簡単に述べたものであり、同時に原稿例、およびスタイルファイルの
使用例である。なお、スタイルファイルのこの部分は「概要」を記述するため
に用意されているが、概要は予稿集原稿の必須項目とはしないので、使用しな
い場合は削除してかまわない。
}
%--------卒業研究予稿集用スタイルファイルのパラメタ ここまで--------

\begin{document}
\maketitle

%%%%%%%%%%%%%%%%%%%%%%%%%%%%%%%%%%%%%%%%%%%%%%%%%%%%%%%%%%%%%%%%
\section{予稿集原稿のフォーマット}
%%%%%%%%%%%%%%%%%%%%%%%%%%%%%%%%%%%%%%%%%%%%%%%%%%%%%%%%%%%%%%%%
本稿のフォーマットが、基本的に卒業研究報告予稿集原稿のフォーマットになっ
ているので、参考にすること。下記に必須の条件を示す。
\begin{itemize}
\item 原稿はA4サイズのPDFとする。最大2ページとし、ページ数の超過は認め
ない。
\item 発表題目と発表者氏名を、原稿PDFのメタデータのTitle、Authorに入れ
ること。
\item 指導教員から指示される発表番号を1ページ目の左肩に書くこと。発表
番号は「会場名--セッション番号--セッション内の発表番号」の形式(例:本
稿はA--2--1)とする。
\item タイトルとして1ページ目の上部に、「発表題目」、「発表者氏名(研
究室名)」を書くこと。
\end{itemize}

以下は必須ではなく、参考程度である。
\begin{itemize}
\item ページ番号は「発表番号--予稿ごとのページ番号」の形式(例:このペー
ジはA--2--1--1)とし、各ページの下部に付けることを推奨する。
\item ページ余白は、上下20mm、左右15mmを目安とする。発表番号やページ番
号は余白領域に記載して構わない。
\item 本文の文字のサイズは9.5ポイント以上を目安とする。本文以外(例え
ば、図表中の文字)についてはこの限りでない。
\item 上記以外のフォーマットは特には指定しない。したがって、段組も本稿
は二段組だが一段組でも問題ないし、本稿のようなレイアウトでの「概要」は
必ずしも必要ではない。
\end{itemize}

%%%%%%%%%%%%%%%%%%%%%%%%%%%%%%%%%%%%%%%%%%%%%%%%%%%%%%%%%%%%%%%%
\section{予稿集原稿用スタイルファイル}
%%%%%%%%%%%%%%%%%%%%%%%%%%%%%%%%%%%%%%%%%%%%%%%%%%%%%%%%%%%%%%%%

%p{\LaTeX}ユーザは、従来のクラスファイル\cite{oldclass}が使える。
本スタイルファイルは、p{\LaTeX}以外の{\LaTeX}エンジン({pdf\LaTeX},
{Xe\LaTeX}, {Lua\LaTeX}等の)ユーザを対象に設計したものである。
サポート外(obsolete)であるが、p{\LaTeX}でも一応動作する。
推奨エンジンは{Lua\LaTeX}とする。

\subsection{documentclassのオプション}
本稿ソースsample.texの冒頭の\verb+\documentclass+で指定しているオプショ
ンについて解説する。本スタイルファイルは、日本語用に注意深く作られた
bxjsarticleクラスを使用することを想定している。下記のオプションは、
bxjsarticleクラスのオプションである。

\begin{description}
\item[autodetect-engine] これを設定しておくと、{\LaTeX}エンジンが自動
判定され、{\LaTeX}エンジンごとに異なる設定がある程度自動化される。
\item[dvi=dvipdfmx] p{\LaTeX}用の設定。p{\LaTeX}を使わないのであれば、
削除して構わない。
\item[ja=standard] 日本語に関して、標準的な設定にする。
\item[twocolumn] 二段組にする。
\item[jbase=13.35Q] 和文フォントのサイズを13.35級(約9.5ポイント)に設
定する。
\end{description}

\subsection{タイトル}
\label{sec:title}
\verb+\maketitle+コマンドにより生成される領域を便宜上タイトルと呼ぶ。
\verb+\title+で発表題目を、\verb+\author+で発表者氏名を設定する。
本スタイルファイルでは、
\verb+\maketitle+の定義を変更して、タイトル要素に以下の項目を追加して
ある。これらをプリアンブルで設定し\verb+\maketitle+すれば、本稿のよう
なタイトルが作成される。
\begin{description}
\item[jcpnumber] 発表番号を設定する。\TeX では「\verb+-+」(-)と
「\verb+--+」(--)は区別するので注意すること。本スタイルファイルは
「\verb+--+」を使うことを前提に作られている。
\item[jcplaboratory] 発表者の研究室名を設定する。
\item[jcpabstract] 概要を設定する。本稿のように、題目や著者に続いて
「概要」が中央上部に表示される。なお、概要が不要な場合は設定しないこと
でタイトルから削除できる。
\end{description}

おまけ機能として、下記も追加した。卒業研究報告予稿集以外の場面に応用で
きるかもしれない。
\begin{description}
\item[jcpsubtitle] 題目の下に付加情報を追加できる。太字。中央配置。
\item[jcpunderauthor] 著者の下に付加情報を追加できる。普通字。左寄
せ。
\end{description}
  
\subsection{ページ番号}
ページ番号に、\verb+jcpnumber+(\ref{sec:title}節参照)で設定した発表
番号が付加される。

\subsection{ページ余白}
本スタイルファイルでは、ページ余白は、上下20mm、左右15mmに設定してある。
変更したいときは、Bxjsclsで用意されている\verb+\setpagelayout+コマンド
を使用するとよい。

%%%%%%%%%%%%%%%%%%%%%%%%%%%%%%%%%%%%%%%%%%%%%%%%%%%%%%%%%%%%%%%%
\section{{\LaTeX}のちょっとしたコツ}
%%%%%%%%%%%%%%%%%%%%%%%%%%%%%%%%%%%%%%%%%%%%%%%%%%%%%%%%%%%%%%%%

{\LaTeX}の基本的な使い方は、巷の本\cite{OkuKuro}や情報サイト
\cite{texwiki}を参照せよ。本稿では、{\LaTeX}の基本的な使い方は割愛する。

{\topsep=0pt
予稿の題目と著者を、生成されるPDFのメタデータのTitle、Authorに入れるに
は、
\begin{verbatim}
\usepackage[pdfusetitle]{hyperref}
\end{verbatim}
をプリアンブルに入れる。

行間を詰めたいときは、
\begin{verbatim}
\renewcommand{\baselinestretch}{0.9}
\end{verbatim}
をプリアンブルに入れる。値を0に近づけるほど行間が詰まる。見苦しくなる
ので、詰めすぎ注意。

本文と数式のフォントをTimes互換フォントにしたいときは、
\begin{verbatim}
\usepackage{newtxtext}
\usepackage[varg]{newtxmath}
\end{verbatim}
をプリアンブルに入れる。
オプションの「\texttt{varg}」は、数式のgのフォントを数学で伝統的な$g$にする。
\textit{g}との違いに注意。
なお、これらのusepackageの宣言は、AMS(数学)関係のパッケージのusepackageよりも後に置くこと。
前に置くと、AMS関係のパッケージをusepackageしたところでエラーになるので注意。

itemize, enumerate, description環境に追加される行間をなくすには、
\begin{verbatim}
\usepackage{enumitem}
\setlist{topsep=0pt,parsep=0pt,partopsep=0pt,
itemsep=0pt}
\end{verbatim}
をプリアンブルに入れる。
}

%%%%%%%%%%%%%%%%%%%%%%%%%%%%%%%%%%%%%%%%%%%%%%%%%%%%%%%%%%%%%%%%
\section{むすび}
%%%%%%%%%%%%%%%%%%%%%%%%%%%%%%%%%%%%%%%%%%%%%%%%%%%%%%%%%%%%%%%%
本稿では卒業研究報告予稿集原稿のフォーマット、および予稿集原稿用
{\LaTeX}スタイルファイル(jcource-proceeding.sty)の使用法を簡単に述べ
た。なお、予稿集原稿はフォーマットさえ守られていれば、どのワープロソフ
ト(Microsoft Word等)を用いて作成してもかまわない。また、{\LaTeX}で作
成する場合でも、必ずしも本スタイルファイルを利用する必要はない(が、本
スタイルファイルを利用するのが手軽である)。

%%%%%%%%%%%%%%%%%%%%%%%%%%%%%%%%%%%%%%%%%%%%%%%%%%%%%%%%%%%%%%%%
% References
%%%%%%%%%%%%%%%%%%%%%%%%%%%%%%%%%%%%%%%%%%%%%%%%%%%%%%%%%%%%%%%%

\begin{thebibliography}{9}% more than 9 --> 99 / less than 10 --> 9
% \bibitem{oldclass}
% 卒業研究発表会サイト, 
% 従来の卒業研究報告予稿集用クラスファイル(p{\LaTeX}専用), 
% \url{https://blue0.an.cis.iwate-u.ac.jp/WebSite/Misc/Graduation/format.html}
\bibitem{OkuKuro}
奥村晴彦, 黒木裕介, {\LaTeXe}美文書作成入門, 技術評論社, (数年に一度改定される)
\bibitem{texwiki}
日本語{\TeX}開発コミュニティ, {\TeX} Wiki, \url{https://texwiki.texjp.org/}
\end{thebibliography}

\end{document}
